\documentclass[12pt,a4paper]{extarticle}
\usepackage[top=2cm, bottom=2cm, left=3cm, right=1.5cm]{geometry}
\usepackage{hyperref}
\usepackage[utf8]{inputenc}
\usepackage{indentfirst}

\usepackage[nottoc, numbib]{tocbibind} % Библиография в содержании (+numbib for numeration)

\usepackage{floatrow}

\usepackage[tableposition=top]{caption}
\usepackage{subcaption}
\usepackage{floatrow}
\floatsetup[table]{capposition=top}
\usepackage{graphicx}
\graphicspath{./graphics/}
\usepackage{float}
\usepackage{wrapfig}

%%%% Установки для размера шрифта 14 pt %%%%
%% Формирование переменных и констант для сравнения (один раз для всех подключаемых файлов)%%
%% должно располагаться до вызова пакета fontspec или polyglossia, потому что они сбивают его работу
\newlength{\curtextsize}
\newlength{\bigtextsize}
\setlength{\bigtextsize}{13.9pt}

\usepackage{polyglossia}
\usepackage{fontspec}

\setmainlanguage[babelshorthands=true]{russian}
\setotherlanguage{english}
\setotherlanguage{greek}

\defaultfontfeatures{Ligatures={TeX},Renderer=Basic}
\setmainfont{Noto Serif} %% задаёт основной шрифт документа
\setsansfont{Noto Serif} %% задаёт шрифт без засечек
\setmonofont{Noto Serif} %% задаёт моноширинный шрифт

\defaultfontfeatures{Ligatures=TeX,Mapping=tex-text,Renderer=Basic}
\defaultfontfeatures{Ligatures=TeX}

\usepackage{amsmath}
\usepackage{amssymb}
\usepackage{setspace}
\onehalfspacing%
\usepackage{tabularx}
\usepackage[table,xcdraw]{xcolor}

\newcommand{\header}{\fontsize{14pt}{21pt}\bf}
\newcommand{\headerl}{\fontsize{12pt}{18pt}\bf}

\usepackage{tocloft}
\renewcommand{\cftsecleader}{\cftdotfill{\cftdotsep}}
\renewcommand{\cftpartleader}{\cftdotfill{\cftdotsep}}
\renewcommand{\cfttoctitlefont}{\header}
\renewcommand{\cftpartfont}{\normalfont}
\renewcommand{\cftsecfont}{\normalfont}
\renewcommand{\cftsecpagefont}{\normalfont}
\renewcommand{\cftpartfont}{\normalfont}
\setlength{\cftbeforesecskip}{1pt}

\usepackage{fancyhdr} % пакет для установки колонтитулов
\pagestyle{fancy} % смена стиля оформления страниц
\fancyhf{} % очистка текущих значений
\fancyhead[R]{\thepage} % установка верхнего колонтитула
%\fancyfoot[R]{\thepage}	% установка нижнего колонтитула
\renewcommand{\headrulewidth}{0bp} % убрать разделительную линию

\pretolerance=5000
\tolerance=9000
\emergencystretch=0pt
\righthyphenmin=4
\lefthyphenmin=4
%----------------------
%	\renewcommand\large{\@setfontsize\large{14}{21}}
%	\renewcommand\Large{\@setfontsize\Large{14}{14}}


\makeatletter
\renewcommand\section{\@startsection{section}{1}{\z@}%
	{-3.5ex \@plus -1ex \@minus -.2ex}%
	{1.1ex \@plus.2ex}%
	{\normalfont\normalfont\bfseries}}% from \Large
\renewcommand\subsection{\@startsection{subsection}{2}{\z@}%
	{-3.25ex\@plus -1ex \@minus -.2ex}%
	{1.1ex \@plus .2ex}%
	{\normalfont\normalfont\bfseries}}% from \large
\renewcommand\subsubsection{\@startsection{subsubsection}{3}{\z@}%
	{-3.25ex\@plus -1ex \@minus -.2ex}%
	{1.1ex \@plus .2ex}%
	{\normalfont\normalfont\bfseries}}% from \normalsize
\makeatother

\captionsetup{font=small, format=plain, labelsep=period, singlelinecheck=off,
	justification=centering}

\parindent=0.5cm
\makeatletter
\renewcommand\@biblabel[1]{#1.}
\makeatother

\let\OLDthebibliography\thebibliography\renewcommand\thebibliography[1]{
	\OLDthebibliography{#1}
	\setlength{\parskip}{0pt}
	\setlength{\itemsep}{0pt plus 0.3ex}
}

\makeatletter
\renewcommand{\paragraph}{%
	\@startsection{paragraph}{4}%
	{\z@}{1ex \@plus 1ex \@minus .2ex}{-1em}%
	{\normalfont\normalsize\bfseries}%
}
\makeatother


\title{Описание программы расчета удельной проводимости полупроводников}
\date{18 декабря 2018 г.}

\begin{document}

\maketitle
\section{Запуск программы}
Исполняемы файл находится в корневой директории и называется \textit{CarrierDensity.exe} (что не совсем соответствует функционалу программы, но так сложилось исторически).

\section{Пользовательский интерфейс}
Программа представляет из себя окно графиком и набором элементов для задания материала и ввода пользовательских данных: уровня энергии доноров \textit{Ed}, концентрации доноров \textit{Nd0}, уровня энергии акцепторов \textit{Ea}, концентрации акцепторов \textit{Na0} и диапазона температуры \textit{T}. Размерность вводимых данных указана непосредственно около поля ввода.

В программе предусмотрена возможность построения графиков зависимости различных величин от температуры: концентрация электронов, концентрация дырок, концентрация заряженных доноров, концентрация заряженных акцепторов, подвижность электронов, подвижность дырок, удельная проводимость.

Переключение между отображаемыми графиками осуществляется путем выбора соответствующего пункта в меню \textit{"Вид"}.

В меню \textit{"Вид"} также присутствует возможность установки отображения графиков в логарифмическом масштабе по оси Y.

Сохранить и загрузить ранее сохраненные графики можно при помощи соответствующих пунктов меню \textit{"Файл"}.

\section{Формат данных}
Все энергии измеряются от потолка валентной зоны (в том числе \textit{Ed} и \textit{Ea}). Размерности указаны в подписях к осям и полям ввода данных.

Пользовательские данные задаются экспоненциальной записью: число представляется в виде мантиссы и экспоненты с указанием знака последней, разделенных английской буквой \textit{e}. Например, $1,6e+19 = 1,6 \cdot 10^{19}$

\section{Пользовательские материалы}
Все внутренние параметры материалов указаны в файле \textit{presets.ini}, находящемся в корневой директории с программой.

Для корректного задания материала, пользователь должен указать следующие величины: ширина запрещенной зоны (\textit{Eg}); эффективная масса электронов (\textit{me}); эффективная масса дырок (\textit{mh}); 2 константы, необходимые для вычисления подвижности электронов (\textit{ae} и \textit{be}); 2 константы, необходимые для вычисления подвижности дырок (\textit{ah} и \textit{bh}).

\section{Описание формата сохранения}
Все графики сохраняются в текстовом виде. В начале файла находится заголовок с внутренними параметрами и параметрами, введенными пользователем.

В заголовке содержаться следующие параметры (соответственно): ширина запрещенной зоны; эффективная масса электронов; эффективная масса дырок; 2 константы, необходимые для вычисления подвижности электронов; 2 константы, необходимые для вычисления подвижности дырок; уровень энергии доноров; концентрация доноров; уровен энергии акцепторов: концентрация акцепторов; название материала.

После заголовка следуют 7 колонок с данными графиков: температура, подвижность электронов, подвижность дырок, концентрация заряженных доноров, концентрация заряженных акцепторов, концентрация электронов, концентрация дырок, удельная проводимость.

\section{Известные ошибки и особенности работы}
При наличии в данных отрицательных или равных нулю величин, график в логарифмическом масштабе не строится.

Иногда при попытке стереть введенные пользователем данные путем нажатия клавиши \textit{Backspace} удаление не происходит (при попытке стереть степень экспоненты). В такие моменты следует выделить все данные в текущем поле, стереть их и набрать заного.


\end{document}